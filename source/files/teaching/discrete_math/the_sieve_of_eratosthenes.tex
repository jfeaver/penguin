
\documentclass{article}

\begin{document}

\pagestyle{empty}

\begin{center}
\large{\textbf{The Sieve of Eratosthenes}}
\end{center}

\noindent The Sieve of Eratosthenes is a method for finding a list of all of the prime numbers less than or equal to a certain integer. Here we will use the sieve to find all primes less than 100, but you could do this for any number. This worksheet is not to be turned in, it is for you to keep. Make sure you hold on to this - this list will be very useful for homework assignments and worksheets on prime numbers.

\vspace{0.2in}

\noindent Follow the steps below. If you do the steps correctly, the circled numbers will be all of the prime numbers less than 100.
\begin{enumerate}
\item Cross out 1 - it is not a prime number!
\item Circle 2 then cross out all of the other numbers which are a multiple of 2. 
\item Circle the next number which is not crossed out. Then cross out all other multiples of this number.
\item Repeat step 3 until all numbers less than or equal to 10 are either crossed out or circled.
\item Circle all remaining numbers - they are all prime!
\end{enumerate}

\vspace{0.1in}

\noindent I had you stop at 10 and circle all of the remaining numbers and claim they are all prime. This works because 10 is the square root of 100; in the Sieve of Eratosthenes you can always stop and circle all remaining numbers once you reach the square root (rounded down to the nearest integer) of your target number. We will go over why this is true in class, because square roots can come in handy when determining if a number is prime or not! \newline

\vspace{0.2in}

\noindent 1 \hspace{0.355in} 2 \hspace{0.355in} 3 \hspace{0.355in} 4 \hspace{0.35in} 5 \hspace{0.35in} 6 \hspace{0.35in} 7 \hspace{0.35in} 8 \hspace{0.35in} 9 \hspace{0.35in} 10 \newline
11 \hspace{0.3in}  12 \hspace{0.3in} 13 \hspace{0.3in} 14 \hspace{0.3in} 15 \hspace{0.3in} 16 \hspace{0.3in} 17 \hspace{0.3in} 18 \hspace{0.3in} 19 \hspace{0.3in} 20 \newline
21 \hspace{0.3in}  22 \hspace{0.3in}  23 \hspace{0.3in}  24 \hspace{0.3in}  25 \hspace{0.3in}  26 \hspace{0.3in}  27 \hspace{0.3in}  28 \hspace{0.3in}  29 \hspace{0.3in} 30 \newline
31 \hspace{0.3in} 32 \hspace{0.3in} 33 \hspace{0.3in} 34 \hspace{0.3in} 35 \hspace{0.3in} 36 \hspace{0.3in} 37 \hspace{0.3in} 38 \hspace{0.3in} 39 \hspace{0.3in} 40 \newline
41 \hspace{0.3in} 42 \hspace{0.3in} 43 \hspace{0.3in} 44 \hspace{0.3in} 45 \hspace{0.3in} 46 \hspace{0.3in} 47 \hspace{0.3in} 48 \hspace{0.3in} 49 \hspace{0.3in} 50 \newline
51 \hspace{0.3in} 52 \hspace{0.3in} 53 \hspace{0.3in} 54 \hspace{0.3in} 55 \hspace{0.3in} 56 \hspace{0.3in} 57 \hspace{0.3in} 58 \hspace{0.3in} 59 \hspace{0.3in} 60 \newline
61 \hspace{0.3in} 62 \hspace{0.3in} 63 \hspace{0.3in} 64 \hspace{0.3in} 65 \hspace{0.3in} 66 \hspace{0.3in} 67 \hspace{0.3in} 68 \hspace{0.3in} 69 \hspace{0.3in} 70 \newline
71 \hspace{0.3in} 72 \hspace{0.3in} 73 \hspace{0.3in} 74 \hspace{0.3in} 75 \hspace{0.3in} 76 \hspace{0.3in} 77 \hspace{0.3in} 78 \hspace{0.3in} 79 \hspace{0.3in} 80 \newline
81 \hspace{0.3in} 82 \hspace{0.3in} 83 \hspace{0.3in} 84 \hspace{0.3in} 85 \hspace{0.3in} 86 \hspace{0.3in} 87 \hspace{0.3in} 88 \hspace{0.3in} 89 \hspace{0.3in} 90 \newline
91 \hspace{0.3in} 92 \hspace{0.3in} 93 \hspace{0.3in} 94 \hspace{0.35in} 95 \hspace{0.31in} 96 \hspace{0.3in} 97 \hspace{0.3in} 98 \hspace{0.3in} 99 \hspace{0.24in} 100 \newline

\end{document}


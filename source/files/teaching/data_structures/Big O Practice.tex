\documentclass{article}

\usepackage{amsthm}
\usepackage{amssymb}
\usepackage{amsmath}
\usepackage{amsthm}
\usepackage{url}
\usepackage{enumerate}
\usepackage{fullpage}
\usepackage{bm}
\usepackage{booktabs}
\usepackage{graphicx}
\usepackage[all]{xy}

\newcommand{\QQ}{\mathbb{Q}}
\newcommand{\ZZ}{\mathbb{Z}}
\newcommand{\RR}{\mathbb{R}}
\newcommand{\CC}{\mathbb{C}}
\newcommand{\p}{\mathfrak{p}}
\newcommand{\q}{\mathfrak{q}}
\newcommand{\B}{\mathfrak{B}}
\newcommand{\cO}{\mathcal{O}}
\newcommand{\cN}{\mathcal{N}}
\DeclareMathOperator{\Gal}{Gal}

\urlstyle{sf}
\newtheorem{thm}{Theorem}[section]
\newtheorem{lem}[thm]{Lemma}
\newtheorem{prop}[thm]{Proposition}
\newtheorem{cor}[thm]{Corollary}
\theoremstyle{definition}
\newtheorem{definition}[thm]{Definition}
\newtheorem{rmk}[thm]{Remark}
\numberwithin{equation}{section}

\pagestyle{empty}

\begin{document}

\noindent\textbf{Big-O Notation Practice}\newline This is an ungraded class activity and you may not have time to answer all of the questions. You have until approximately 2:35-2:40 to work on this.

\vspace{0.4in}

\noindent Review of limits - evaluate the following:

\vspace{0.2in}

$\lim_{n\to\infty}\frac{n^{10}-1000n^4+12}{3n^{10}-n^9}$

\vfill

$\lim_{n\to\infty}\frac{1}{e^n}$

\vfill

$\lim_{n\to\infty}\frac{n!}{n^n}$

\vfill

$\lim_{n\to\infty}\frac{n^4+n^3-36}{n+8}$

\vfill

\noindent \textbf{Instructions:} Answer the remainder of the questions on this worksheet \textbf{twice} - once using the limit definition of big-O notation, and once using the inequality definition.

\vspace{0.2in}

An algorithm takes $f(n)=n^7+12$ operations to execute. How should you describe this run-time using big-O notation? Prove that your answer is correct using each definition.

\vspace{0.4in}

\vfill

\newpage

Prove that the function from the last question is $O(8n^{11})$. Even though this fits the definition of big-O, what is wrong with using $8n^{11}$ to describe the run-time of your algorithm?

\vfill

Prove that the function $f$ is not $O(n+\sqrt{n})$.

\vfill

\end{document}


\documentclass{article}
\usepackage{fullpage}
\pagestyle{empty}

\begin{document}

\textwidth = 6.5in
\textheight = 9.7in
\oddsidemargin = 0.0in
\evensidemargin = 0.0in
\topmargin = 0.0in
\headheight = -0.45in
\headsep = 0.0in
\parskip = 0.2in
\parindent = 0.0in

\pagestyle{empty}

\textbf{Names:} \rule{6.1in}{0.01in}

\vspace{0.3in}

\noindent\rule{6.7in}{0.01in}

\centerline{Math 1110\hfill{\bf Four Fours Worksheet}\hfill Fall 2012}

\vspace{0.2in}

\noindent Credit: This worksheet was inspired heavily by \textbf{Dr. Eric Stade} at The University of Colorado.

\vspace{0.2in}

\noindent Express each of the integers from 1 to 9 using exactly four fours, together with addition, subtraction, multiplication, division, and parenthesis. As an example, I did the first one for you:

$1=4\div4\times4\div4$

\vspace{0.1in}

\noindent Now it's your turn. Don't forget to put parenthesis where needed. 

2=\rule{2in}{0.01in}

3=\rule{2in}{0.01in}

4=\rule{2in}{0.01in}

5=\rule{2in}{0.01in}

6=\rule{2in}{0.01in}

7=\rule{2in}{0.01in}

8=\rule{2in}{0.01in}

9=\rule{2in}{0.01in}

\newpage

\noindent Using the method and rules above, but now allowing juxtaposition (that is: it's OK to use a 44, but this takes up two of your fours). Express ANY THREE of the numbers 10,11,12,13,14,15,16 (whichever three you want) using four fours.

\rule{1in}{0.01in}=\rule{2in}{0.01in}

\rule{1in}{0.01in}=\rule{2in}{0.01in}

\rule{1in}{0.01in}=\rule{2in}{0.01in}

\end{document}
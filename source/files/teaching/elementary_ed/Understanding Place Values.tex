
\documentclass{article}

\usepackage{fullpage}
\pagestyle{empty}


\begin{document}

\noindent\textbf{Name:}

\vspace{0.2in}

\begin{center}
\Large{Place Values}
\end{center}

\vspace{0.2in}
\noindent One difference between our number system and Roman numerals is that we have \textit{place values}.  You've heard of these - ones, tens, hundreds, etc.
This may seem basic to you, but we are going to review place values before talking about other number systems because
\textbf{understanding place values in our number system will help you to learn numbers in other bases and different number systems more easily.}

\vspace{0.2in}

\noindent Because our numbers are in base ten, they can be broken up as follows:  
\[342 = 3\cdot100 + 4\cdot 10 +2\cdot1,\]
splitting the number into hundreds ($10^2$), tens ($10^1$) and ones ($10^0$).

\vspace{1in}

\noindent Below, you are going to practice splitting up numbers in this way. I did the first two for you:
\begin{enumerate}
\item $561=5\cdot100 + 6\cdot 10 +1$
\vspace{0.3in}
\item $0.909=9\cdot(1/10) + 0\cdot (1/100) +9\cdot(1/1000)$
\vspace{0.3in}
\item 3,333 =
\vspace{0.3in}
\item 20,000 =
\vspace{0.3in}
\item 8.41 =
\vspace{0.3in}
\item 24,689 =
\end{enumerate}
   
\newpage

\noindent Now we're going to review and talk about exponents a little bit, because they can be used to write place values in an easier way. Remember, and exponent means to multiply a number by itself that many times. For example:
\[2^3=2\cdot2\cdot2=8.\]

\vspace{0.2in}

\noindent Compute the following exponential expressions:
\begin{enumerate}
\item $3^2 = 3\cdot3=9$
\vspace{0.3in}
\item $1^5=1\cdot1\cdot1\cdot1\cdot1=1$
\vspace{0.3in}
\item $5^2 =$
\vspace{0.3in}
\item $6^1 = 6$
\vspace{0.3in}
\item $111^1=$
\vspace{0.3in}
\item $10^1=$
\vspace{0.3in}
\item $10^2=$
\vspace{0.3in}
\item $10^3=$
\vspace{0.3in}
\item $10^4=$
\vspace{0.3in}
\item $10^5=$
\vspace{0.3in}
\item $10^{-6}=$
\vspace{0.3in}
\item $2^0=$
\vspace{0.3in}
\item $1,000,000,003^0=$
\end{enumerate}

\newpage

\noindent Okay, now you're ready to write your list of numbers, using exponents, to show the base 10 structure:

\vspace{0.2in}

\begin{enumerate}
\item $561=5\cdot10^2 + 6\cdot 10^1 +1\cdot 10^0$
\vspace{0.3in}
\item $0.909=9\cdot10^{-1} + 0\cdot 10^{-2} +9\cdot 10^{-3}$
\vspace{0.3in}
\item 3,333 = 
\vspace{0.3in}
\item 20,000 =
\vspace{0.3in}
\item 8.41 =
\vspace{0.3in}
\item 24,689 =
\end{enumerate}

\end{document}
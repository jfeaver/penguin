

\documentclass{article}
\usepackage{fullpage}
\pagestyle{empty}

\begin{document}

\textwidth = 6.5in
\textheight = 9.7in
\oddsidemargin = 0.0in
\evensidemargin = 0.0in
\topmargin = 0.0in
\headheight = -0.45in
\headsep = 0.0in
\parskip = 0.2in
\parindent = 0.0in

\pagestyle{empty}

\textbf{Names:} \rule{6.1in}{0.01in}

\vspace{0.3in}

\noindent\rule{6.7in}{0.01in}

\centerline{Math 1110\hfill{\bf Palindromic Numbers}\hfill Fall 2012}

\vspace{0.2in}

\begin{enumerate}

\item A palindrome is a word or phrase that reads the same both forward and backward. Some examples are the phrases, \textit{Madam, I'm Adam} and \textit{No sir away! A papaya war is on!}, and the words \textit{kayak} and \textit{racecar}. Given this information, explain what you think a palindromic number is:

\vspace{2in}

\item In the space below, list three four-digit palindromic numbers. Then check whether or not each number is divisible by 11.

\vspace{2in}

\item Are all four-digit palindromic numbers divisible by 11? Explain!

\newpage

\item Are all five-digit palindromes divisible by 11?\newline
If your answer is yes, explain why. If your answer is no, give two examples of five-digit palindromic numbers which are not divisible by 11.

\vspace{2.5in}

\item If $n$ is any even integer, what do you know about $n$-digit palindromic numbers?

\vspace{2.5in}

\item Use the information above to find two numbers which are larger than 1,000,000,000 and are divisible by 11. Use the divisibility test for 11 to check that your answers work.

\newpage

\item A palindromic pair is a pair of words or phrases, where each is the other spelled backward. Some examples are \textit{(stressed, desserts)} and \textit{(warts, straw)}. In the space below, write down three pairs of palindromic numbers.

\vspace{2in}

\item In a palindromic pair, if one number is divisible by three, then so is the other. Why is this true?

\vspace{3in}

\item If one number in a palindromic pair is divisible by 9, is the other number also divisible by 9? Explain you answer!

\newpage

\item Could you ever have a palindromic pair where both numbers are divisible by 10? Why or why not?



\end{enumerate}



\end{document}
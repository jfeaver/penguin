

\documentclass{article}
\usepackage{fullpage}
\pagestyle{empty}

\begin{document}

\textwidth = 6.5in
\textheight = 9.7in
\oddsidemargin = 0.0in
\evensidemargin = 0.0in
\topmargin = 0.0in
\headheight = -0.45in
\headsep = 0.0in
\parskip = 0.2in
\parindent = 0.0in

\pagestyle{empty}

\textbf{Names:} \rule{6.1in}{0.01in}

\vspace{0.3in}

\noindent\rule{6.7in}{0.01in}

\centerline{Math 1110\hfill{\bf Scientific Notation}\hfill Fall 2012}

\vspace{0.2in}

\begin{enumerate}

\item Assume you have a piece of paper which has a thickness of $1\times 10^{-7}$ km. If you were to fold it in half, how thick would the folded paper be?

\vspace{0.5in}

\item If you were to fold the paper in half again (so that it has been folded twice), how thick would it be?

\vspace{0.5in}

\item If you were to fold the paper in half a third time, it would be the same thickness as a stack of \rule{0.5in}{0.01in} pieces of paper, and the thickness of this stack would be \rule{0.5in}{0.01in} km.

\vspace{0.7in}

\item Now, using scientific notation, fill in the chart (it is continued on the back of this page):

\end{enumerate}

Number of Folds  \hspace{0.3in}  Equivalent to ? Pieces of Paper \hspace{0.3in}  Thickness (km) \newline

\vspace{0.1in}

0 \hspace{1.3in}  1 \hspace{2.2in} $1\times 10^{-7}$

\vspace{0.1in}

1 \hspace{1.3in}  2 \hspace{2.2in} $2\times 10^{-7}$

\vspace{0.1in}

2 \hspace{1.3in}  \rule{0.5in}{0.01in} \hspace{1.7in} \rule{0.5in}{0.01in}

\vspace{0.1in}

3 \hspace{1.3in}  \rule{0.5in}{0.01in} \hspace{1.7in} \rule{0.5in}{0.01in}

\vspace{0.1in}

4 \hspace{1.3in}  \rule{0.5in}{0.01in} \hspace{1.7in} \rule{0.5in}{0.01in}

\vspace{0.1in}

5 \hspace{1.3in}  \rule{0.5in}{0.01in} \hspace{1.7in} \rule{0.5in}{0.01in}

\vspace{0.1in}

6 \hspace{1.3in}  \rule{0.5in}{0.01in} \hspace{1.7in} \rule{0.5in}{0.01in}

\vspace{0.1in}

7 \hspace{1.3in}  \rule{0.5in}{0.01in} \hspace{1.7in} \rule{0.5in}{0.01in}

\vspace{0.1in}

8 \hspace{1.3in}  \rule{0.5in}{0.01in} \hspace{1.7in} \rule{0.5in}{0.01in}

\vspace{0.1in}

9 \hspace{1.3in}  \rule{0.5in}{0.01in} \hspace{1.7in} \rule{0.5in}{0.01in}

\vspace{0.1in}

10 \hspace{1.25in}  \rule{0.5in}{0.01in} \hspace{1.7in} \rule{0.5in}{0.01in}

\vspace{0.1in}

11 \hspace{1.25in}  \rule{0.5in}{0.01in} \hspace{1.7in} \rule{0.5in}{0.01in}

\vspace{0.1in}

12 \hspace{1.25in}  \rule{0.5in}{0.01in} \hspace{1.7in} \rule{0.5in}{0.01in}

\vspace{.3in}

To help you visualize this, after only 12 folds, the thickness of your paper will be a little less than 1 and a half feet, which is thousands of times thicker than the piece of paper that you started out with. Using this information, I want your group to guess (without calculating) how thick your paper would be if you were able to fold it the following number of times. Choose your answers from the list below and write the corresponding letter in the blank.

\vspace{0.1in}

\begin{enumerate}

\item If folded 14 times the paper would be about \rule{1in}{0.01in} km thick. 

\item If folded 30 times the paper would be about \rule{1in}{0.01in} km thick.

\item If folded 50 times the paper would be about \rule{1in}{0.01in} km thick.

\item If folded 100 times the paper would be about \rule{1in}{0.01in} km thick.

\vspace{0.2in}

\textbf{Answer Choices}

\begin{enumerate}

\item $1.6\times10^{-3}$ km, the height of an average human

\item $7\times10^{-3}$km, the length of the average male killer whale

\item 8.848km, the height of Mount Everest

\item 10km, the length of the Bolder Boulder

\item $1.074\times10^2$ km, the distance from the ground to the outer limits of the atmosphere

\item $1.125\times10^8$ km, the distance to the sun.

\item $4.731\times10^9$km, the distance from Mercury to Pluto

\item $1.268\times10^{23}$ km, the estimated radius of the known universe

\item $9.872\times10^{23}$km

\end{enumerate}

\end{enumerate}

\newpage

Answers:

If folded 14 times the paper would be about $1.6\times10^{-3}$ km thick. About the height of the average person.

If folded 30 times the paper would be about $1.074\times10^2$ km thick. The distance from the ground to the outer limits of the atmosphere.

If folded 50 times the paper would be about $1.125\times10^8$ km thick. The distance to the sun.

If folded 100 times the paper would be about $1.268\times10^{23}$ km thick. The estimated radius of the known universe.



\end{document}